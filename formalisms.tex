\documentclass{article}
\usepackage{amsmath, amssymb}
\usepackage{geometry}
\geometry{margin=1in}

\title{Z-Curvature Prime Detection and GR Triangle Formalism}
\author{Your Name}
\date{}

\begin{document}

\maketitle

\section*{1. Z-Curvature and Prime Filtering}

Let \( n \in \mathbb{N} \), \( n > 1 \), and define:

\begin{itemize}
    \item \( d(n) \): number of positive divisors of \( n \)
    \item \( \log(n) \): natural logarithm
    \item \( e \): Euler’s number
    \item \( \mathbb{P} \): set of prime numbers
\end{itemize}

\subsection*{1.1 Z-Curvature Function}

\[
Z_{\text{curv}}(n) := \frac{d(n) \cdot \log(n)}{e^2}
\]

\subsection*{1.2 Curvature Threshold}

\[
\Theta(n) := \max \left( 3.5,\ \max_{\substack{p \in \mathbb{P} \\ p < n}} Z_{\text{curv}}(p) \right)
\]

\subsection*{1.3 Z-Prime Filter}

\[
\mathcal{Z}_\text{prime}(n) := 
\begin{cases}
\text{True}, & \text{if } Z_{\text{curv}}(n) \leq \Theta(n) \\
\text{False}, & \text{otherwise}
\end{cases}
\]

\subsection*{1.4 Z-Based Prime Classifier}

\[
\text{IsPrime}_{\mathcal{Z}}(n) := 
\begin{cases}
\text{False}, & \text{if } Z_{\text{curv}}(n) > \Theta(n) \\
\text{IsPrime}(n), & \text{otherwise}
\end{cases}
\]

\section*{2. GR-Inspired Z-Triangles}

For a given pair of successive primes \( p_k, p_{k+1} \), define:

\begin{align*}
    \Delta n &= p_{k+1} - p_k \\
    \log_p &= \log(p_k) \\
    g &= \frac{\Delta n}{\log_p} \quad \text{(normalized gap)} \\
    C_k &= Z_{\text{curv}}(p_k) \\
    R_k &= Z_{\text{res}}(p_k) := \left( \frac{p_k \bmod \log_p}{e} \right) \cdot d(p_k) \\
    \theta_k &= \tan^{-1}\left( \frac{R_k}{C_k} \right) \quad \text{(Z-angle)} \\
    \| \vec{Z}_k \| &= \sqrt{C_k^2 + R_k^2}
\end{align*}

---

\subsection*{2.1 GR Triangle 1: Gravitational Lensing Analogue}

This triangle models curvature, gap, and angular deviation.

\[
\boxed{
\textbf{Triangle 1: } 
\left\{
\begin{aligned}
A &= C_k \quad \text{(mass-induced curvature)} \\
B &= g \quad \text{(path: normalized gap)} \\
C &= \frac{\theta_k}{90^\circ} \quad \text{(bending)}
\end{aligned}
\right.
}
\]

Interpretation: Mass bends space over a gap producing angular distortion.

---

\subsection*{2.2 GR Triangle 2: Metric Tensor Analogue}

This triangle links raw prime gap with the Z-vector magnitude:

\[
\boxed{
\textbf{Triangle 2: } 
\left\{
\begin{aligned}
A &= \Delta n \quad \text{(gap)} \\
B &= 1 \quad \text{(unit baseline)} \\
C &= \| \vec{Z}_k \| \quad \text{(field vector)}
\end{aligned}
\right.
}
\]

Interpretation: Magnitude of the Z-vector expresses metric field strength between primes.

---

\subsection*{2.3 GR Triangle 3: Frame-Dragging Analogue}

This triangle captures the differential “drag” in the Z-field between two primes.

Let:

\[
\begin{aligned}
\Delta C &= |Z_{\text{curv}}(p_{k+1}) - Z_{\text{curv}}(p_k)| \\
\Delta R &= |Z_{\text{res}}(p_{k+1}) - Z_{\text{res}}(p_k)| \\
g &= \text{(normalized gap as above)}
\end{aligned}
\]

Then:

\[
\boxed{
\textbf{Triangle 3: } 
\left\{
\begin{aligned}
A &= \Delta C \quad \text{(curvature change)} \\
B &= \Delta R \quad \text{(resonance change)} \\
C &= g \quad \text{(normalized gap)}
\end{aligned}
\right.
}
\]

Interpretation: Frame-dragging is modeled as Z-field deviation between successive primes.

---

\section*{3. Ontological Notes}

The set of primes \( \mathbb{P} \) may be viewed as low-curvature fixed points in a relativistic Z-field. The triangles formalize field geometry and detect rapid transitions in curvature and resonance across discrete time (\( n \)).

\end{document}